\documentclass[aspectratio=169]{beamer}
\usetheme{Madrid}
\usecolortheme{default}
\usepackage[utf8]{inputenc}
\usepackage[vietnamese]{babel}
\usepackage{amsmath}
\usepackage{amsfonts}
\usepackage{amssymb}
\usepackage{graphicx}
\usepackage{hyperref}
\usepackage{listings}
\usepackage{xcolor}

\title[Quantum Computing Research]{Quantum Computing: Recent Advances}
\subtitle{Three Research Papers on Quantum GANs and Protein Folding}
\author{Presentation}
\date{\today}

\begin{document}

\frame{\titlepage}

\begin{frame}{Nội dung trình bày}
\tableofcontents
\end{frame}

% ============================================================================
% PAPER 1: Quantum GANs on Silicon Photonic Chip
% ============================================================================
\section{Paper 1: Quantum GANs trên Silicon Photonic Chip}

\begin{frame}{Paper 1: Tổng quan}
\textbf{Tiêu đề:} Quantum Generative Adversarial Networks in a Silicon Photonic Chip with Maximum Expressibility

\vspace{0.3cm}
\textbf{Tác giả:} Haoran Ma, Liao Ye, Fanjie Ruan, et al. (Zhejiang University)

\vspace{0.3cm}
\textbf{Nguồn:} arXiv:2404.05921v1

\vspace{0.3cm}
\textbf{Lĩnh vực:} Quantum Machine Learning, Silicon Photonics, GANs
\end{frame}

\begin{frame}{Động lực nghiên cứu}
\textbf{Vấn đề:}
\begin{itemize}
    \item Quantum GANs có tiềm năng lợi thế hàm mũ so với classical GANs
    \item Cần nền tảng phần cứng có khả năng biểu diễn cao (high expressibility)
    \item Các chip photonic trước đây bị giới hạn về khả năng tạo trạng thái lượng tử
\end{itemize}

\vspace{0.3cm}
\textbf{Giải pháp:}
\begin{itemize}
    \item Thiết kế chip silicon photonic 2-qubit có thể tạo \textbf{bất kỳ trạng thái thuần 2-qubit nào}
    \item Thực thi các phép toán Controlled-Unitary (CU) tùy ý
    \item Kết hợp AMZI (Asymmetrical MZI) và frequency post-selection
\end{itemize}
\end{frame}

\begin{frame}{Kiến trúc chip}
\textbf{Thành phần chính:}
\begin{enumerate}
    \item \textbf{Nguồn photon:} 2 spiral waveguides tạo photon pairs qua SFWM
    \item \textbf{AMZI:} Điều chỉnh biên độ entangled states
    \item \textbf{Controlled-Unitary operations:} Tạo trạng thái 2-qubit tùy ý
    \item \textbf{Single-qubit gates:} State tomography và tính toán
\end{enumerate}

\vspace{0.3cm}
\textbf{Đặc điểm kỹ thuật:}
\begin{itemize}
    \item Kích thước: $3mm \times 0.8mm$
    \item 14 phase shifters (PS) điều khiển nhiệt-quang
    \item Bước sóng: Signal 1555.75nm, Idler 1546.12nm
    \item Coupling loss: $\sim 4.5$ dB
\end{itemize}
\end{frame}

\begin{frame}{AMZI với Frequency Post-selection}
\textbf{Nguyên lý hoạt động:}

Ma trận truyền AMZI:
\[
U_{AMZI} = e^{j(\beta_{s,i}\Delta l + \phi)} \begin{bmatrix}
\sin((\beta_{s,i}\Delta l + \phi)/2) & \cos((\beta_{s,i}\Delta l + \phi)/2) \\
\cos((\beta_{s,i}\Delta l + \phi)/2) & -\sin((\beta_{s,i}\Delta l + \phi)/2)
\end{bmatrix}
\]

\vspace{0.3cm}
\textbf{Coincidence rate:}
\[
C(\phi) \propto C_{max}\sin^4(\phi/2)
\]

\vspace{0.3cm}
\textbf{Trạng thái tạo ra:}
\[
|\psi_0\rangle = e^{i\theta_s}\frac{\sqrt{C_1(\phi_1)}|0_s\rangle|0_i\rangle + \sqrt{C_2(\phi_2)}|1_s\rangle|1_i\rangle}{A}
\]

với góc quay: $\tan(\phi) = \sqrt{C_2(\phi_2)/C_1(\phi_1)}$
\end{frame}

\begin{frame}{Khả năng tạo trạng thái tùy ý}
\textbf{Quantum circuit tương đương:}

Kết hợp $R_y(\phi)$ gate và 2 unitary $\hat{U}$ và $\hat{V}$ có thể tạo bất kỳ trạng thái 2-qubit nào:

\[
|\psi_2\rangle = \sqrt{p_0}|00\rangle + \sqrt{p_1}|01\rangle + \sqrt{p_2}|10\rangle + \sqrt{p_3}|11\rangle
\]

với $p_0 + p_1 + p_2 + p_3 = 1$

\vspace{0.3cm}
\textbf{Ý nghĩa:}
\begin{itemize}
    \item Hệ số trước mỗi basis state hoàn toàn tùy ý
    \item Maximum expressibility cho quantum GANs
    \item Vượt trội so với các công trình trước [32-36]
\end{itemize}
\end{frame}

\begin{frame}{Thí nghiệm 1: Học trạng thái Single-qubit}
\textbf{Mô hình PQ-GAN (Pure Quantum GAN):}
\begin{itemize}
    \item Generator G: Tạo trạng thái lượng tử $\rho(\theta_g)$
    \item Discriminator D: Đo quantum measurement $M(\theta_d)$
\end{itemize}

\vspace{0.3cm}
\textbf{Hàm tối ưu:}
\[
\min_{\theta_g} \max_{\theta_d} \text{tr}[M(\theta_d)\rho(\theta_g)] - \text{tr}[\sigma M] = 0
\]

\vspace{0.3cm}
\textbf{Kết quả:}
\begin{itemize}
    \item \textbf{Pure state:} Fidelity \textbf{99.41\%} (trạng thái mục tiêu: $(|0\rangle + |1\rangle)/\sqrt{2}$)
    \item \textbf{Mixed state:} Fidelity \textbf{98.39\%} (trạng thái: $0.7|0\rangle\langle 0| + 0.3|1\rangle\langle 1|$)
    \item Huấn luyện: 200 epochs, D:G ratio = 3:1
    \item Learning rates: $\eta_G = 0.02$, $\eta_D = 0.1$
\end{itemize}
\end{frame}

\begin{frame}{Thí nghiệm 2: Load Classical Distribution}
\textbf{Kiến trúc HQC-GAN (Hybrid Quantum-Classical):}
\begin{itemize}
    \item Quantum Generator: PQC với 3 tham số trainable
    \item Classical Critic: Fully-connected NN (4-5-3-1)
    \item Loss function: Wasserstein distance với gradient penalty
\end{itemize}

\vspace{0.3cm}
\textbf{Hàm tối ưu (WGAN-GP):}
\[
\min_{\theta_g} \max_{\theta_c} D_{p_\theta}(G(\theta_g)) - D_{p_\theta}(\hat{x}) + \lambda \mathbb{E}_{p_\theta}[||\nabla_\theta D_{p_\theta}(\hat{x})||_2 - 1]^2
\]

\vspace{0.3cm}
\textbf{Distributions được học:}
\begin{enumerate}
    \item Normal: $X \sim N(\mu=1.5, \sigma=1)$
    \item Log-normal: $X \sim LN(\mu=0.5, \sigma=0.5)$
    \item Bimodal: Superposition của 2 Gaussians
\end{enumerate}
\end{frame}

\begin{frame}{Kết quả Load Distribution}
\textbf{Hiệu suất:}
\begin{itemize}
    \item KLD (Kullback-Leibler Divergence) < 0.05 cho cả 3 distributions
    \item Huấn luyện: 500 epochs, 5 rounds với initialization khác nhau
    \item Critic:Generator ratio = 3:1
    \item Learning rates: $\eta_G = 0.08$, $\eta_C = 0.1$
\end{itemize}

\vspace{0.3cm}
\textbf{Ưu điểm:}
\begin{itemize}
    \item WGAN-GP khắc phục mode collapse và vanishing gradients
    \item Quantum circuit depth thấp (chỉ 3 tham số)
    \item Data encoding vào basis state probabilities: $\vec{p} = [p_0, p_1, p_2, p_3]^T$
\end{itemize}
\end{frame}

\begin{frame}{Thí nghiệm 3: Tạo ảnh MNIST nén}
\textbf{Hybrid Generator mới:}
\begin{itemize}
    \item Classical NN ($2 \times 2$ với Leaky ReLU) + Quantum PQC
    \item Mục đích: Đưa nonlinearity vào quantum GANs
    \item Input: Noise vector $z \sim U[0,1]$
    \item Output: MNIST digits $2 \times 2$ sau PCA
\end{itemize}

\vspace{0.3cm}
\textbf{Preprocessing:}
\begin{itemize}
    \item MNIST $28 \times 28$ $\rightarrow$ PCA $\rightarrow$ 3 dimensions
    \item Augment: $\vec{x}' = [\vec{x}^T, 0.5]$
    \item Map: $p_i = x'_i / \sum_{j=0}^{3} x'_j$
\end{itemize}

\vspace{0.3cm}
\textbf{Gradient computation:}
\[
\frac{\partial \theta_g}{\partial t} = \frac{L(\theta_g + \epsilon) - L(\theta_g - \epsilon)}{2\epsilon}
\]
(Finite difference method với $\epsilon = 0.02$)
\end{frame}

\begin{frame}{Kết quả MNIST Generation}
\textbf{Thành công:}
\begin{itemize}
    \item Tạo được hình ảnh cho tất cả 10 digits (0-9)
    \item KLD và critic loss hội tụ về 0
    \item Batch size: 5, 200 epochs
    \item Learning rates: $\eta_{NN} = 0.02$, $\eta_{PQC} = 0.08$, $\eta_C = 0.02$
\end{itemize}

\vspace{0.3cm}
\textbf{Đổi mới:}
\begin{itemize}
    \item \textbf{Lần đầu tiên} quantum photonic chip học mixed states
    \item \textbf{Lần đầu tiên} tạo compressed images với silicon photonic
    \item Hybrid generator khác với cách tiếp cận hiện có: giữ info trong quantum state thay vì classical post-processing
    \item Có thể tích hợp vào quantum circuits phức tạp hơn
\end{itemize}
\end{frame}

\begin{frame}{Paper 1: Đóng góp và Hạn chế}
\textbf{Đóng góp chính:}
\begin{enumerate}
    \item Chip silicon photonic có maximum expressibility (tạo mọi trạng thái 2-qubit)
    \item 3 ứng dụng quantum GANs thành công trên phần cứng thực
    \item Hybrid generator với classical NN để thêm nonlinearity
    \item Kết quả SOTA cho quantum photonic GANs
\end{enumerate}

\vspace{0.3cm}
\textbf{Hạn chế:}
\begin{itemize}
    \item Chỉ 4 dimensions (2 qubits) - giới hạn ứng dụng
    \item Success rate bị ảnh hưởng bởi post-selection
    \item Requires cryogenic cooling cho SNSPDs
\end{itemize}

\vspace{0.3cm}
\textbf{Triển vọng:}
\begin{itemize}
    \item Mở rộng lên high-dimensional encoding
    \item Kết hợp với patched GAN cho large-scale data
    \item Ứng dụng trong quantum state distribution loading
\end{itemize}
\end{frame}

% ============================================================================
% PAPER 2: Protein Folding with Trapped-Ion QC
% ============================================================================
\section{Paper 2: Protein Folding với Trapped-Ion Quantum Computer}

\begin{frame}{Paper 2: Tổng quan}
\textbf{Tiêu đề:} Protein folding with an all-to-all trapped-ion quantum computer

\vspace{0.3cm}
\textbf{Tác giả:} Sebastián V. Romero, et al. (Kipu Quantum \& IonQ Inc.)

\vspace{0.3cm}
\textbf{Nguồn:} arXiv:2506.07866v2

\vspace{0.3cm}
\textbf{Lĩnh vực:} Quantum Optimization, Protein Folding, Trapped-Ion Computing
\end{frame}

\begin{frame}{Bài toán Protein Folding}
\textbf{Tầm quan trọng:}
\begin{itemize}
    \item Cấu trúc 3D protein quyết định chức năng sinh học
    \item Hiểu folding $\rightarrow$ drug design, disease treatment
    \item Thuật toán cổ điển: AlphaFold2 (AI-based) rất thành công
\end{itemize}

\vspace{0.3cm}
\textbf{Tiếp cận lượng tử:}
\begin{itemize}
    \item Map protein folding $\rightarrow$ ground-state search
    \item Higher-Order Unconstrained Binary Optimization (HUBO)
    \item Lattice model: Tetrahedral lattice (4 directions/residue)
\end{itemize}

\vspace{0.3cm}
\textbf{Challenges:}
\begin{itemize}
    \item Exponential growth của search space
    \item Dense coupling giữa các amino acids
    \item Hardware noise và limited connectivity
\end{itemize}
\end{frame}

\begin{frame}{BF-DCQO Algorithm}
\textbf{Bias-Field Digitized Counterdiabatic Quantum Optimization:}

\vspace{0.2cm}
\textbf{Ưu điểm:}
\begin{itemize}
    \item Non-variational $\rightarrow$ tránh barren plateaus
    \item Phù hợp với dense HUBO problems
    \item Tận dụng all-to-all connectivity của trapped-ion
\end{itemize}

\vspace{0.3cm}
\textbf{Hamiltonian:}
\[
H_{total} = \lambda_c H_c + \lambda_g H_g + \lambda_d H_d + \lambda_i H_i
\]

\begin{itemize}
    \item $H_c$: Chirality constraints
    \item $H_g$: Geometric constraints
    \item $H_d$: Steric/distance constraints
    \item $H_i$: Miyazawa-Jernigan interactions
\end{itemize}
\end{frame}

\begin{frame}{Two-Stage Architecture}
\textbf{Tại sao cần 2 stages?}
\begin{itemize}
    \item Hardware noise ảnh hưởng đến measurement accuracy
    \item Tách energy estimation và structural decoding
\end{itemize}

\vspace{0.3cm}
\textbf{Stage 1: Energy Estimation}
\begin{itemize}
    \item Chạy BF-DCQO để tìm ground state
    \item Đo năng lượng quantum system
    \item Output: Optimal parameters
\end{itemize}

\vspace{0.3cm}
\textbf{Stage 2: Structural Decoding}
\begin{itemize}
    \item Fix parameters từ Stage 1
    \item Chạy với shots cao hơn để decode cấu trúc
    \item Map bitstring $\rightarrow$ 3D coordinates
\end{itemize}
\end{frame}

\begin{frame}{Hardware: IonQ Trapped-Ion}
\textbf{Đặc điểm:}
\begin{itemize}
    \item All-to-all connectivity (fully connected graph)
    \item 36 qubits available
    \item High fidelity 2-qubit gates
    \item Longer coherence times so với superconducting
\end{itemize}

\vspace{0.3cm}
\textbf{Circuit Pruning Strategies:}

\textbf{1. Soft Pruning:}
\begin{itemize}
    \item Chọn 1000 DCQO solutions ngẫu nhiên
    \item Đánh giá bằng local search
    \item Giữ top-5 solutions
\end{itemize}

\textbf{2. Hard Pruning (better):}
\begin{itemize}
    \item Giới hạn entangling gates ở hàng trăm
    \item Giảm circuit depth và noise
    \item Consistently outperforms soft pruning
\end{itemize}
\end{frame}

\begin{frame}{Kết quả: Protein Sequences}
\textbf{Tested proteins:}
\begin{enumerate}
    \item \texttt{GYDPETGTWG} (10 amino acids)
    \item \texttt{QPPGGSKVILF} (11 amino acids)
    \item \texttt{WTFGQGTKVEIK} (12 amino acids - \textbf{33 qubits})
\end{enumerate}

\vspace{0.3cm}
\textbf{Achievements:}
\begin{itemize}
    \item \textbf{Optimal solutions} cho cả 3 sequences
    \item \textbf{Largest quantum protein folding} implementation to date
    \item Energy correlation với conformational energy rất tốt
\end{itemize}

\vspace{0.3cm}
\textbf{Energy scaling:}
\begin{itemize}
    \item 5 residues $\rightarrow$ 14 residues: Minimum energy tăng \textbf{230,000\%}
    \item Thể hiện exponential growth của energy landscape
\end{itemize}
\end{frame}

\begin{frame}{Kết quả: MAX 4-SAT}
\textbf{Testing robustness với combinatorial optimization:}

\vspace{0.3cm}
\textbf{Setup:}
\begin{itemize}
    \item MAX 4-SAT instances at computational phase transition
    \item Clause-to-variable ratio $\sim 9.7$
    \item Kích thước: 24-36 qubits
\end{itemize}

\vspace{0.3cm}
\textbf{Results:}
\begin{itemize}
    \item \textbf{Optimal solutions achieved} cho tất cả instances
    \item Consistent performance across different problem sizes
    \item Validates algorithm effectiveness beyond protein folding
\end{itemize}
\end{frame}

\begin{frame}{Kết quả: Spin-Glass Problems}
\textbf{Fully connected spin-glass (36 qubits):}

\vspace{0.3cm}
\textbf{Test cases:}
\begin{itemize}
    \item 3 random instances với all-to-all coupling
    \item Dense interaction graph
\end{itemize}

\vspace{0.3cm}
\textbf{Performance:}
\begin{itemize}
    \item \textbf{2 out of 3} instances: Exact ground state found
    \item 1 instance: Near-optimal solution
    \item Demonstrates synergy between:
    \begin{itemize}
        \item Non-variational optimization
        \item All-to-all connectivity hardware
    \end{itemize}
\end{itemize}
\end{frame}

\begin{frame}{Paper 2: So sánh và Triển vọng}
\textbf{So với Classical/AI approaches:}
\begin{itemize}
    \item AlphaFold2: Data-driven, cần large training set
    \item Quantum: Physics-based energy minimization
    \item Quantum có thể complement AI methods
\end{itemize}

\vspace{0.3cm}
\textbf{Pathway to Quantum Advantage:}
\begin{enumerate}
    \item Trapped-ion scalability đang được cải thiện
    \item BF-DCQO tránh được barren plateau problem
    \item Dense HUBO problems với industrial relevance
\end{enumerate}

\vspace{0.3cm}
\textbf{Future directions:}
\begin{itemize}
    \item Longer protein sequences (>15 residues)
    \item Improved lattice models (off-lattice)
    \item Integration với experimental validation
    \item Drug discovery applications
\end{itemize}
\end{frame}

% ============================================================================
% PAPER 3: Protein Structure Prediction Framework
% ============================================================================
\section{Paper 3: Framework Dự đoán Cấu trúc Protein}

\begin{frame}{Paper 3: Tổng quan}
\textbf{Tiêu đề:} Prediction of Protein Three-dimensional Structures via a Hardware-Executable Quantum Computing Framework

\vspace{0.3cm}
\textbf{Tác giả:} Yuqi Zhang, et al. (Kent State, Cleveland Clinic, Harvard)

\vspace{0.3cm}
\textbf{Nguồn:} arXiv:2506.22677

\vspace{0.3cm}
\textbf{Lĩnh vực:} Quantum Computing, Structural Biology, Drug Discovery
\end{frame}

\begin{frame}{Động lực: Beyond AlphaFold}
\textbf{AlphaFold3 limitations:}
\begin{itemize}
    \item Data-driven approach
    \item "Information trap" cho short peptides
    \item Limited capacity với fragments < 10 residues
    \item Không trực tiếp optimize physical energy
\end{itemize}

\vspace{0.3cm}
\textbf{Quantum advantage hypothesis:}
\begin{itemize}
    \item Physics-based: Trực tiếp minimize Hamiltonian
    \item Higher theoretical reliability cho peptide fragments
    \item Better cho active site prediction (drug binding)
\end{itemize}

\vspace{0.3cm}
\textbf{Goal:}
\begin{itemize}
    \item End-to-end executable pipeline trên utility-level quantum hardware
    \item Validation với therapeutic proteins
    \item Benchmarking vs AlphaFold3
\end{itemize}
\end{frame}

\begin{frame}{Framework Architecture}
\textbf{Complete Pipeline:}

\begin{enumerate}
    \item \textbf{Problem Formulation}
    \begin{itemize}
        \item Tetrahedral lattice encoding
        \item Hamiltonian construction: $H_t = \lambda_c H_c + \lambda_g H_g + \lambda_d H_d + \lambda_i H_i$
    \end{itemize}

    \item \textbf{VQE Optimization}
    \begin{itemize}
        \item EfficientSU2 ansatz
        \item COBYLA optimizer
        \item 2,000 shots, $\geq 200$ iterations
    \end{itemize}

    \item \textbf{Two-Stage Execution}
    \begin{itemize}
        \item Stage 1: Energy estimation
        \item Stage 2: Fixed-parameter measurement (20,000 shots)
    \end{itemize}

    \item \textbf{Post-processing}
    \begin{itemize}
        \item Atom completion
        \item Charge neutralization
        \item PDB file generation
    \end{itemize}
\end{enumerate}
\end{frame}

\begin{frame}{Hardware: IBM Quantum}
\textbf{Platform:}
\begin{itemize}
    \item IBM-Cleveland Clinic 127-qubit processor
    \item Eagle r3 architecture
    \item Heavy-hex topology (limited connectivity)
\end{itemize}

\vspace{0.3cm}
\textbf{Encoding scheme:}
\begin{itemize}
    \item Sparse Pauli operators
    \item Amino acid connectivity trong tetrahedral lattice
    \item Constraints: Chirality, geometry, distance, interactions
\end{itemize}

\vspace{0.3cm}
\textbf{Computational efficiency:}
\begin{itemize}
    \item Average: $\sim 10$ seconds/iteration
    \item 73.53\% time trên quantum end
    \item Rest: Classical optimization overhead
\end{itemize}
\end{frame}

\begin{frame}{Evaluation: Metrics}
\textbf{Geometric accuracy:}
\[
\text{RMSD} = \sqrt{\frac{1}{N}\sum_{i=1}^{N} ||r_i^{pred} - r_i^{ref}||^2}
\]

\vspace{0.3cm}
\textbf{Functional consistency:}
\begin{itemize}
    \item Molecular docking với AutoDock Vina
    \item Binding affinity (kcal/mol)
    \item Lower is better (stronger binding)
\end{itemize}

\vspace{0.3cm}
\textbf{Test set:}
\begin{itemize}
    \item 23 protein fragments từ PDB
    \item Lengths: 5-10 residues
    \item 7 với therapeutic potential
\end{itemize}
\end{frame}

\begin{frame}{Kết quả: vs AlphaFold3}
\textbf{Overall performance (23 fragments):}

\vspace{0.3cm}
\begin{table}
\centering
\begin{tabular}{lcc}
\hline
\textbf{Metric} & \textbf{Quantum Method} & \textbf{AlphaFold3} \\
\hline
Average RMSD (Å) & \textbf{3.33} & 3.87 \\
Average Binding Affinity (kcal/mol) & \textbf{-4.38} & -4.00 \\
\hline
\end{tabular}
\end{table}

\vspace{0.3cm}
\textbf{Superiority rates:}
\begin{itemize}
    \item RMSD: \textbf{18 out of 23} cases lower
    \item Binding affinity: \textbf{21 out of 23} cases better
\end{itemize}

\vspace{0.3cm}
\textbf{Statistical significance:}
\begin{itemize}
    \item Clear advantage cho short peptides
    \item Especially good for active site regions
\end{itemize}
\end{frame}

\begin{frame}{Kết quả: Therapeutic Proteins}
\textbf{Validation với 7 therapeutic targets:}

\vspace{0.2cm}
\begin{enumerate}
    \item \textbf{6mu3} - Anti-HIV-1 Fab 2G12
    \item \textbf{3ans} - Human soluble epoxide hydrolase
    \item \textbf{1a9m} - HIV-1 protease G48H
    \item \textbf{1qin} - Lactoylglutathione lyase
    \item \textbf{3b26} - HSP 90-alpha
    \item \textbf{1fkn} - Beta-Secretase BACE1 (Alzheimer's)
    \item \textbf{2xxx} - Glutamate receptor GluK2
\end{enumerate}

\vspace{0.3cm}
\textbf{Success:}
\begin{itemize}
    \item All structures predicted successfully
    \item Suitable for molecular docking
    \item Demonstrates feasibility cho drug discovery
\end{itemize}
\end{frame}

\begin{frame}{Energy-Structure Correlation}
\textbf{Key finding:}
\begin{itemize}
    \item Positive correlation giữa quantum system energy và docking affinity
    \item Lower quantum energy $\rightarrow$ Better binding affinity
    \item Validates physics-based approach
\end{itemize}

\vspace{0.3cm}
\textbf{Implications:}
\begin{itemize}
    \item Quantum method captures relevant molecular interactions
    \item Energy minimization meaningful cho structural biology
    \item Not just mathematical optimization - physically grounded
\end{itemize}

\vspace{0.3cm}
\textbf{Advantage over AI:}
\begin{itemize}
    \item AI learns patterns, quantum solves physics
    \item More reliable cho novel sequences
    \item Less dependent on training data distribution
\end{itemize}
\end{frame}

\begin{frame}{Scalability: Sliding Window}
\textbf{Challenge:}
\begin{itemize}
    \item Long proteins > 10 residues
    \item Exponential qubit requirements
\end{itemize}

\vspace{0.3cm}
\textbf{Solution - Sliding Window approach:}
\begin{itemize}
    \item Window size: 7 residues
    \item Stride: 1 residue
    \item Overlap và merge fragments
\end{itemize}

\vspace{0.3cm}
\textbf{Demonstration:}
\begin{itemize}
    \item Full-length A$\beta$42 (Alzheimer's peptide)
    \item Successfully predicted từ overlapping windows
    \item Enables handling arbitrary length sequences
\end{itemize}

\vspace{0.3cm}
\textbf{Future improvement:}
\begin{itemize}
    \item Better merging algorithms
    \item Adaptive window sizes
    \item Multi-scale optimization
\end{itemize}
\end{frame}

\begin{frame}{Paper 3: Đóng góp chính}
\textbf{Scientific contributions:}
\begin{enumerate}
    \item \textbf{First} complete hardware-executable pipeline cho protein structure
    \item \textbf{First} validation trên utility-level quantum processors
    \item \textbf{Outperforms} AlphaFold3 cho short peptides
    \item Direct application to drug discovery
\end{enumerate}

\vspace{0.3cm}
\textbf{Technical innovations:}
\begin{itemize}
    \item Two-stage architecture for noise mitigation
    \item Sliding-window scalability
    \item Post-processing for docking compatibility
    \item Comprehensive benchmarking methodology
\end{itemize}

\vspace{0.3cm}
\textbf{Practical impact:}
\begin{itemize}
    \item Blueprint cho domain-specific quantum applications
    \item Demonstrates utility-level quantum computing feasibility
    \item Real-world therapeutic protein validation
\end{itemize}
\end{frame}

% ============================================================================
% COMPARISON & CONCLUSION
% ============================================================================
\section{So sánh và Kết luận}

\begin{frame}{So sánh 3 Papers}
\begin{table}
\scriptsize
\centering
\begin{tabular}{|p{2.5cm}|p{3.5cm}|p{3.5cm}|p{3.5cm}|}
\hline
\textbf{Aspect} & \textbf{Paper 1} & \textbf{Paper 2} & \textbf{Paper 3} \\
\hline
Hardware & Silicon Photonic & Trapped-Ion (IonQ) & Superconducting (IBM) \\
\hline
Qubits & 2 qubits & Up to 36 qubits & Up to 127 qubits \\
\hline
Algorithm & Quantum GANs & BF-DCQO & VQE \\
\hline
Application & Machine Learning & Protein Folding & Structure Prediction \\
\hline
Key Strength & Max expressibility & All-to-all connectivity & End-to-end pipeline \\
\hline
Main Result & 99.41\% fidelity & 12 AA folded & Beats AlphaFold3 \\
\hline
\end{tabular}
\end{table}

\vspace{0.3cm}
\textbf{Common themes:}
\begin{itemize}
    \item Demonstrating quantum utility on real hardware
    \item Addressing NISQ-era challenges (noise, limited qubits)
    \item Application-driven research
\end{itemize}
\end{frame}

\begin{frame}{Hardware Platforms Comparison}
\textbf{Silicon Photonic (Paper 1):}
\begin{itemize}
    \item[+] Room temperature operation
    \item[+] Low noise, high precision
    \item[+] Naturally suited for quantum communication
    \item[-] Limited scalability (currently 2 qubits)
    \item[-] Post-selection reduces success rate
\end{itemize}

\vspace{0.2cm}
\textbf{Trapped-Ion (Paper 2):}
\begin{itemize}
    \item[+] All-to-all connectivity
    \item[+] High gate fidelities
    \item[+] Long coherence times
    \item[-] Slower gates
    \item[-] Moderate scalability challenges
\end{itemize}

\vspace{0.2cm}
\textbf{Superconducting (Paper 3):}
\begin{itemize}
    \item[+] Most scalable (100+ qubits)
    \item[+] Fast gates
    \item[-] Limited connectivity (heavy-hex)
    \item[-] Higher noise, shorter coherence
\end{itemize}
\end{frame}

\begin{frame}{Quantum GANs vs Protein Folding}
\textbf{Different approaches to quantum advantage:}

\vspace{0.3cm}
\textbf{Quantum GANs (Paper 1):}
\begin{itemize}
    \item Exploit quantum superposition và expressibility
    \item Generator tạo quantum states
    \item Hybrid quantum-classical learning
    \item Focus: Machine learning applications
\end{itemize}

\vspace{0.3cm}
\textbf{Protein Folding (Papers 2 \& 3):}
\begin{itemize}
    \item Ground-state energy minimization
    \item Map biological problem $\rightarrow$ Hamiltonian
    \item Physics-based approach
    \item Focus: Scientific computing applications
\end{itemize}

\vspace{0.3cm}
\textbf{Complementary directions:}
\begin{itemize}
    \item GANs: Generative models, creative tasks
    \item Folding: Optimization, structure discovery
    \item Both demonstrate NISQ-era quantum utility
\end{itemize}
\end{frame}

\begin{frame}{Challenges và Limitations}
\textbf{Common challenges across all papers:}

\vspace{0.3cm}
\begin{enumerate}
    \item \textbf{Hardware noise}
    \begin{itemize}
        \item Mitigation: Two-stage architectures, error mitigation
        \item Still limits problem sizes
    \end{itemize}

    \item \textbf{Limited qubit counts}
    \begin{itemize}
        \item Workarounds: Circuit pruning, sliding windows
        \item Fundamental scaling needed
    \end{itemize}

    \item \textbf{Classical-Quantum interface}
    \begin{itemize}
        \item Measurement overhead
        \item Parameter optimization loops
        \item Data encoding/decoding
    \end{itemize}

    \item \textbf{Validation}
    \begin{itemize}
        \item How to verify quantum advantage?
        \item Need better benchmarks
        \item Classical baselines improving rapidly
    \end{itemize}
\end{enumerate}
\end{frame}

\begin{frame}{Future Directions}
\textbf{Near-term (1-3 years):}
\begin{itemize}
    \item Scale to 50-100 logical qubits với error correction
    \item Better hybrid classical-quantum algorithms
    \item Application-specific quantum processors
    \item Improved noise mitigation techniques
\end{itemize}

\vspace{0.3cm}
\textbf{Medium-term (3-7 years):}
\begin{itemize}
    \item Fault-tolerant quantum computing
    \item Quantum advantage cho practical problems
    \item Integration với AI/ML pipelines
    \item Commercial quantum applications
\end{itemize}

\vspace{0.3cm}
\textbf{Long-term (7+ years):}
\begin{itemize}
    \item Universal quantum computers
    \item Drug discovery revolution
    \item Materials design
    \item Cryptography và security
\end{itemize}
\end{frame}

\begin{frame}{Ý nghĩa thực tiễn}
\textbf{Drug Discovery (Papers 2 \& 3):}
\begin{itemize}
    \item Faster protein structure prediction
    \item Better binding affinity predictions
    \item Novel therapeutic target discovery
    \item Personalized medicine
\end{itemize}

\vspace{0.3cm}
\textbf{Machine Learning (Paper 1):}
\begin{itemize}
    \item Quantum-enhanced generative models
    \item Distribution learning cho finance, physics
    \item Quantum data encoding
    \item Hybrid classical-quantum AI
\end{itemize}

\vspace{0.3cm}
\textbf{Scientific Computing:}
\begin{itemize}
    \item Materials science simulations
    \item Chemical reaction modeling
    \item Optimization problems
    \item Fundamental physics research
\end{itemize}
\end{frame}

\begin{frame}{Kết luận chung}
\textbf{Key takeaways:}

\vspace{0.3cm}
\begin{enumerate}
    \item \textbf{Quantum utility is emerging}
    \begin{itemize}
        \item Real hardware demonstrations
        \item Competitive with/surpassing classical methods
        \item Application-specific advantages
    \end{itemize}

    \item \textbf{Multiple hardware platforms viable}
    \begin{itemize}
        \item Photonic: Precision, low noise
        \item Trapped-ion: Connectivity, fidelity
        \item Superconducting: Scale, speed
    \end{itemize}

    \item \textbf{Hybrid approaches essential}
    \begin{itemize}
        \item Classical-quantum co-design
        \item Noise mitigation strategies
        \item Domain-specific optimizations
    \end{itemize}

    \item \textbf{Path to quantum advantage}
    \begin{itemize}
        \item Focus on specific applications
        \item Leverage quantum strengths
        \item Continuous hardware improvement
    \end{itemize}
\end{enumerate}
\end{frame}

\begin{frame}{Tài liệu tham khảo}
\textbf{Papers discussed:}

\vspace{0.3cm}
\begin{enumerate}
    \item Haoran Ma, et al. "Quantum Generative Adversarial Networks in a Silicon Photonic Chip with Maximum Expressibility." arXiv:2404.05921v1, 2024.

    \vspace{0.2cm}
    \item Sebastián V. Romero, et al. "Protein folding with an all-to-all trapped-ion quantum computer." arXiv:2506.07866v2, 2025.

    \vspace{0.2cm}
    \item Yuqi Zhang, et al. "Prediction of Protein Three-dimensional Structures via a Hardware-Executable Quantum Computing Framework." arXiv:2506.22677, 2025.
\end{enumerate}

\vspace{0.5cm}
\textbf{Additional resources:}
\begin{itemize}
    \item IBM Quantum: \url{https://quantum-computing.ibm.com}
    \item IonQ Platform: \url{https://ionq.com}
    \item Xanadu Photonics: \url{https://xanadu.ai}
\end{itemize}
\end{frame}

\begin{frame}[c]
\centering
\Huge{Cảm ơn!}

\vspace{1cm}
\Large{Questions?}
\end{frame}

\end{document}
